\documentclass[english,12pt,a4paper]{article}
\usepackage[T1]{fontenc}
\usepackage{babel}
\usepackage{amsmath}
\title{Machine Learning}
\author{Fafer77}
\begin{document}
	\maketitle
	\section{Definitions}
	\begin{enumerate}
		\item Measures of central tendency (they describe a central position of your data):
		\begin{enumerate}
			\item Mean
			\item Mode
			\item Median
		\end{enumerate}
		\item Measures of spread (describe how spread out your data is - clumped together or spread far apart):
		\begin{enumerate}
			\item Range
			\item Quartiles
			\item Standard deviation
			\item Variance
		\end{enumerate}
		\item Training set - data used for training
		\item Training example (sample) - A single instance used in machine learning to train a model. It typically consists of input features and, in supervised learning, an associated label or target value.
		\item Accuracy - the ratio of correctly predicted instances to the total number of instances in the dataset. 
		\[
		\text{Accuracy} = \frac{\text{Number of correct predictions}}{\text{Total number of predictions}}
		\]
		\item Classification task - involves predicting a discrete label or category for a given input based on its features e.g 'spam' or 'not spam'
		\item Data mining - analyzing huge amount of data to find some patterns
	\end{enumerate}
\end{document}